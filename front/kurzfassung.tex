\chapter{Kurzfassung}
% Das Verwenden von Containern wird aufgrund deren Leichtgewichtigkeit immer beliebter.

Container sind eine leichtgewichtige Alternative zu virtuellen Maschinen und
werden dadurch immer beliebter. Um Container zu ermöglichen, sind unter Linux
verschiedenste Mechanismen verfügbar. Einerseits sind Mechanismen vorhanden,
welche die vom Container benötigten Systemressourcen isolieren, und
andererseits Mechanismen, welche den Host vor unerlaubten Zugriffen aus dem Container
beschützen. Das Ziel dieser Bachelorarbeit ist es, den Stand der Technik der
vorhandenen Mechanismen zur Isolation von Containern aufzuzeigen. Ebenso werden
die eventuell vorhandenen Probleme geschildert.

Um die Forschungsfragen zu beantworten, werden zuerst die zur Isolation von
Containern verwendeten Linux Kernel Features und anschließend die dazu
verwendeten Linux Security Modules behandelt. Die Funktionalität der
Mechanismen wird mithilfe von Beispielen demonstriert.

Es zeigt sich, dass alle nötigen Mechanismen für eine erfolgreiche Isolation
von Containern vorhanden sind. Die Verwendung von Linux Security Modules ist
ein besonders wichtiger Mechanismus, um Hosts vor einem Container zu
beschützen. Jedoch weisen vorhanden Linux Security Modules Mängel bei
Verwendung mit Containern auf. Aktuelle Forschungen fokussieren sich auf die
vorhanden Mängel und versuchen diese zu beheben.
