\chapter{Abstract}

\begin{english} 
Containers are more lightweight than virtual machines and therefore gaining
popularity. Under Linux there are several mechanisms available to make
containers possible. On one hand there are mechanisms needed to isolate the
necessary system resources and on the other hand there are mechanisms needed to
protect the host from attacks out of the container. 

The goal of this thesis is to show the current state-of-the-art of mechanisms
to successfully isolate containers. To answer the research questions the
necessary Linux Kernel Features are explained first. The second part is the
explanation of the necessary Linux Security Modules. Additionally, the
functionality of the mechanism is demonstrated with examples.

The result of this thesis shows that all necessary mechanisms, to successfully
isolate containers, are available. The use of Linux Security Modules is a very
important mechanism to protect the host from containers. However, existing
Linux Security Modules are not best suited for this use case. Current
development and research is focused on improving usability of Linux Security
Modules with containers.
\end{english}