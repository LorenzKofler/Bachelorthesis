\chapter{Fazit und Ausblick}
\label{cha:Zusammenfassung}
Die zwei wichtigsten Bausteine für Container sind Namespaces und Control
Groups. Die acht verschiedenen Namespaces werden verwendet, um globale
Systemressourcen für die Verwendung mit Containern zu abstrahieren. Zum
Beispiel mittels des PID Namespaces werden die Prozesse des Containers vom
restlichen System isoliert. Durch den erst seit kurzem als sicher gesehenen
User Namespace kann die Verwendung von privilegierten Containern vermieden
werden. Unprivilegierte Container bieten besseren Schutz vor
Privilegien-Eskalation. Control Groups werden verwendet, um den
Ressourcenverbrauch, wie zum Beispiel CPU und Arbeitsspeicher eines Containers
zu limitieren. Durch Namespaces und Control Groups wird der Anschein der
Virtualisierung innerhalb eines Containers erweckt.

Wenn der Linux-Kernel kompromittiert wird, können diese beiden Mechanismen
einfach ausgehebelt werden. Deshalb ist es notwendig, zusätzliche
Schutzmechanismen zu verwenden. Mit Seccomp-BPF können Systemcalls blockiert
werden, um mögliche Auswirkungen von kompromittierten Prozessen zu reduzieren
und die Angriffsoberfläche auf den Linux Kernel zu verringern.
Capabilities teilen die Rechte des Superusers in kleine Einheiten. Durch das
Entfernen von Capabilities können privilegierte Prozesse innerhalb von
Containern weiter eingeschränkt werden, damit wird erreicht, dass diese weniger
Schaden anrichten können. Linux Security Modules wie AppArmor oder SELinux
werden als zusätzlicher Schutzmechanismus verwendet. Diese beiden Mechanismen
erlauben dem*der Systemadministrator*in Sicherheitsrichtlinien zu erzwingen. Damit
kann verhindert werden, dass Container Zugriff auf sensible Dateien haben.

Verbesserungswürdig sind die vorhandenen Linux Security Modules.
Unprivilegierten Containern ist es mit AppArmor und SELinux nicht möglich,
selbstständig benutzerdefinierte Sicherheitsrichtlinien anzuwenden. AppArmor
arbeitet an einer Lösung zu diesem Problem. Ebenso sind Landlock und der
Security Namespace Vorschläge um genanntes Problem zu lösen. Landlock wird in
naher Zukunft Teil des Linux Kernels sein. 

Aus diesen Neuerungen ergibt sich ein weiterer Forschungsbedarf. Mittels einem
Vergleich der neuen Sicherheitsmodule soll evaluiert werden, welche Lösung am
besten für die Verwendung mit Containern geeignet ist. Ebenso eine
Gegenüberstellung von alternativen, nicht Linux basierten, Mechanismen zur
Isolation von Container ist von Relevanz.
