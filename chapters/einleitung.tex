\chapter{Einleitung}
\label{cha:Einleitung}

\section{Ausgangssituation}
Container sind eine leichtgewichtige Alternative zu hypervisorbasierter
Virtualisierung und bieten in beinahe allen Anwendungsfällen gleiche oder
bessere Performance \cite{ContainerPerformance}.
Dadurch wird die Verwendung von Containern in den verschiedensten
Einsatzgebieten immer beliebter \cite{ContainerAdoption}.

Der Unterschied zu hypervisorbasierter Virtualisierung besteht darin, dass
diese die nötige Hardware emulieren. Das Gast{-}, sowie das Host-Betriebssystem
verwenden jeweils einen eigenen Kernel. Container im Gegensatz bieten eine
isolierte Umgebung an, welche den Anschein einer Virtualisierung erweckt. Es
werden bestimmte Prozesse und Ressourcen wie RAM\footnote{Random-Access
Memory}, CPU\footnote{Central Processing Unit} et cetera vom Host und anderen
Containern isoliert. Dabei teilen Container und Host sich den gleichen Kernel. 

\section{Problemstellung}
Bei der Verwendung von Containern entsteht aufgrund des geteilten Kernels ein
Problem.  Damit Container sicher genützt werden können, müssen diese
Mechanismen verwenden, um die Ressourcen des Containers vollständig vom
Hostsystem zu isolieren.  Ebenso müssen jegliche unerlaubte Zugriffe auf das
Hostsystem blockiert werden.

Somit werden unterschiedliche Linux Kernel Features und Linux Security Modules
verwendet, um den Container zu isolieren und das Hostsystem zu schützen.  Genau
genommen sind Mechanismen vorhanden, welche die Isolation von allen
nötigen Ressourcen des Containers ermöglichen und dadurch den Anschein einer 
Virtualisierung innerhalb des Containers erwecken.
Des Weiteren sind Mechanismen
vorhanden, welche das Host-Betriebssystem vor Angriffen aus dem Container
schützen, indem die Angriffsoberfläche auf den Kernel limitiert und der
unberechtigte Zugriff auf Dateien außerhalb des Containers blockiert wird.  

\section{Zielsetzung}
Es wird der Stand der Technik der unter Linux vorhandenen Mechanismen zur
Isolation von Containern aufgezeigt.  Ebenso werden die eventuell vorhandenen
Probleme dieser Mechanismen geschildert. Des Weiteren wird ein Überblick über
den Stand der Forschung gegeben und wie dieser die Probleme des Stands der
Technik lösen kann.

\section{Forschungsfragen}
Was sind der Stand der Technik und der Stand der Forschung von Mechanismen, die
unter Linux zur Verfügung stehen, um einen Container zu isolieren und das
Hostsystem vor diesem zu schützen?

\begin{itemize}
  \item Welche Features bietet der Linux-Kernel zum Isolieren von Containern
    und was sind deren Limits und Probleme?
  \item Welche Linux Security Modules bietet der Linux-Kernel zum Isolieren von
    Containern und was sind deren Limits und Probleme?
  \item Was ist der Stand der Forschung von Mechanismen unter Linux zum
    Isolieren von Containern?
\end{itemize}

\section{Methodik}
Bei dieser Arbeit handelt es sich um eine Literaturstudie. Hierbei wird der
Stand der Technik und der Stand der Forschung von Mechanismen, welche unter
Linux verfügbar sind, aufgezeigt. Hauptsächlich werden dabei Linux Kernel
Features und Linux Security Modules betrachtet. Im Laufe der Arbeit werden die
Probleme dieser Mechanismen genannt. Es wird ein Überblick über den
Stand der Forschung geschaffen und wie dieser die Probleme der Stand der
Technik löst. Schlussendlich wird der Stand der Technik der Mechanismen
und deren Probleme zusammengefasst.

\section{Erwartete Ergebnisse}
Das erwartete Ergebnis ist eine vollständige Sammlung von Mechanismen, welche
dem Stand der Technik entsprechen, um Container unter Linux zu isolieren und
deren Probleme aufzuzeigen. 
